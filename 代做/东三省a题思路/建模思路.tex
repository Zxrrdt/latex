\documentclass[]{article}
\usepackage{ctex}
\usepackage{graphicx}
\usepackage{amsmath}
\usepackage{amssymb}
\graphicspath{{pictures1/}}
\usepackage[linewidth=1pt]{mdframed}
% Title Page
\title{火星探测器着陆控制方案-建模思路}
\author{Sun Quan}


\begin{document}
\maketitle

\section{问题}
	该情景需要解决的问题主要有以下几点:
	
	1.确定探测器着陆时间最短的方案(环绕器与着陆巡视器分离、阻尼伞打开、发动机系统点火等时间,以及发动机系统运行方案);
	
	2.对给定的着陆过程时间,确定消耗能量最少的操控方案;
	
	3.如果希望探测器着陆过程与公开的音像和文字资料一致,如何设计操控方案。
\section{思考方式}
	\subsection{官方情景描述}
	为了清楚火星探测器着陆控制方案应当如何确定,首先应当了解火星探测器的着陆运动形式。天问一号火星探测器的官方报道如下:
	
	\begin{mdframed}
		\begin{center}
			\includegraphics[scale=0.7]{baodao1}
		\end{center}		
	\end{mdframed}

	由题意可知,我们所讨论的探测器着陆时间最短方案,应当研究的过程是自环绕器与着陆巡视器分离开始,至着陆巡视器安全着陆。其中需要我们进行确定的操作为:阻尼伞打开时机、发动机系统点火时机,另外还要确定发动机系统的运行方案(即在发动机运行过程内,发动机应当如何工作)。
	\subsection{基础模型构建}
	涉及到运动学方面问题的,我们一般都会利用微分方程来进行求解。在这道题中,我们只需要合理建立起相应的运动学模型,即可对该题做一个很好的分析(前提条件是我们的微分方程模型构建地十分恰切)。最简单地来说,只需要找到探测器运动路程与时间的关系即可解决第一与第二问,第三问再根据查阅的资料,依照所需结果改变操控方案即可对应模拟,测试出最终的结果
	\subsubsection{运动过程描述}
	首先,为了建立起合适的微分方程模型,我们应当对火星探测器运动过程的每一环节进行合理地描述(在此处利用模型假设可以很好地降低建模难度)。首先我们知道,探测器在整个着陆运动过程中,包含:分离滑行段、攻角配平与升力控制段、伞降控制段、动力减速段、悬停成像段、避障机动段以及触地与主发动机关机环节。
	
	显然,这些环节太过繁琐,对于我们的数学建模而言并不需要考虑到如此详尽(性价比不高且对模型的计算结果并不会产生大的影响),故而我们在此处应当主要考虑几个重要环节。
	
	实际上,在官方题目中也已经给出了,于是我们可以将整个运动过程分为简单的三个部分:\textbf{攻角配平与升力控制段、伞降控制段、动力减速段}(可以按照需求,自己进行假设,只要合理即可)。原因在于避障机动段、触地与主发动机关机环节可以归于动力减速段之中,并且不同的操作方案在这两个环节上是不会发生太多改变的,所以可以不考虑这两个环节对方案的影响。
\subsubsection{微分方程模型建立}
	首先建立微分方程,我们知道在该问题中我们需要涉及的几个关键物理量包括时间、质量、运动路程、运动方向等等。针对整个问题,我们可以构建一个偏微分方程,因为变量不只一个(涉及到探测器的质量变化,比如燃料质量损失、抛弃隔热大底和抛弃背罩等都会影响探测器质量,进而影响到加速度,还可能与火星引力有关)。在这里,我们的思路很显然可以分为两种:
	
\textbf{	1.忽略燃料质量损失(如果燃料质量相对探测器质量很小),然后再根据抛弃隔热大底、打开发动机等分段构建微分方程进行考虑。这样做的好处是,非常的简单,因为质量在每一段运动过程中都将是一个常量;}
	
\textbf{	2.不忽略燃料质量的损失,构建微分方程进行求解计算。当然,同样也是分段建立(非常简单),由于燃料质量的损失是一个连续的过程,实际上这种思路也很简单。}
	
	两种思路,前者会简化到小学生难度,不推荐。
	
	所以,我们只需要根据第二种思路,建立起每一个阶段的微分方程,然后再分别求解并求和即可得到最终的结果。
	
	举一个最简单的示范(所有因素都不考虑,在具体建模时必须要考虑各个因素):对于动力减速段而言,探测器具备一个初速度,然后受惯性其会继续坠落,而此时选定一个时间段开启发动机,则可提供一个加速度使探测器速度减慢。(在我的这个示范中没有考虑其他的受力,要想模型做的好,必须讨论各种细节问题,需要想办法进行改进)
\begin{equation}
	\begin{split}
		-\frac{dx}{dt}&=v_0+at\\
		&=v_0+\frac{F}{m(t)}t\\
		&=v_0+\frac{F_{fuel}-\frac{GMm(t)}{r^2}t}{m(t)}\\
		&=v_0+\frac{F_{fuel}-\frac{GMm(t)}{(R+x)^2}t}{m(t)}
	\end{split}
\end{equation}
	可以看到,这样就得到了在动力减速段的一个简单微分方程。对其进行求解,即可得到动力减速段,时间与着陆器运动路程的关系。但是可以发现,我在此处用的是$m(t)$,原因很简单,正如之前所说,着陆器的质量变化是需要考虑的。在此处,我们可以简单地考虑,如果只是燃料损耗,那么假设燃料损耗为线性损耗,那么可以设质量变化如下
	\begin{equation}
		m=m_0-kt
	\end{equation}
其中,$k$为燃料损耗系数。然后可以代入上式,求得结果。
	当然,这是我举的最简单的例子,事实上,为了让模型有着更好的仿真结果,燃料的损耗可以不用线性函数,具体还想怎么改进需要你们自己来思考了。另外,我上面的公式直接引用了万有引力公式,事实上探测器的下降过程中,路径的变化应该对其所受火星重力的影响微乎其微,故也可以进行简化处理,我就不详细赘述了。
	
	确定完了微分方程后,直接利用数值计算方法(比如\textbf{欧拉法、改进欧拉法以及四阶龙格库塔法})进行求解即可得到结果(如果构建的方程很简单的话,直接手算也可能得到解析解,这里内容有待丰富)。

	最后,类似地考虑其他运动过程即可。注意,在攻角配平与升力控制段应当注重受力分析,多看一些相关论文,但如果实在看不下去可以进行合理简化。按照这一方法,建立起微分方程显然可以根据各条件得到各时刻的速度、加速度以及路程等,所以最后一问也可以求得。

Good Luck.
\end{document}          
